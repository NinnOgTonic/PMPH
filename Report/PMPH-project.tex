%Template by Morten Espensen
%Deffinere kommandoer til om vores information

\documentclass[]{report}
%\usepackage{color}
%\usepackage{alltt}
\usepackage{amsmath}
\usepackage{float}
\usepackage[parfill]{parskip}
\usepackage{lscape}
\usepackage{enumerate}

\usepackage[danish,english]{babel}

\usepackage[paper=A4,pagesize]{typearea}
\usepackage[toc,page]{appendix}

\usepackage[utf8]{inputenc}
\usepackage{fancyhdr}
%\usepackage{boxproof}
%\usepackage{daymonthyear}
\usepackage{stmaryrd}

\usepackage{color}
\usepackage{fancyvrb}
\usepackage[usenames,dvipsnames]{xcolor}
\fvset{frame=single,framesep=1mm,fontfamily=courier,fontsize=\scriptsize,numbers=left,framerule=.3mm,numbersep=1mm,commandchars=\\\{\}}

\usepackage{wallpaper} % For the frontpage background

\usepackage{listings} % KODE SHIT
\lstset{ %
language=PHP,          		    % the language of the code
basicstyle=\footnotesize,       % the size of the fonts that are used for the code
numbers=left,                   % where to put the line-numbers
numberstyle=\footnotesize,      % the size of the fonts that are used for the line-numbers
stepnumber=1,                   % the step between two line-numbers. If it's 1, each line
                                % will be numbered
numbersep=5pt,                  % how far the line-numbers are from the code
backgroundcolor=\color{white},  % choose the background color. You must add \usepackage{color}
showspaces=false,               % show spaces adding particular underscores
showstringspaces=false,         % underline spaces within strings
showtabs=false,                 % show tabs within strings adding particular underscores
frame=single,                   % adds a frame around the code
tabsize=2,                      % sets default tabsize to 2 spaces
captionpos=b,                   % sets the caption-position to bottom
breaklines=true,                % sets automatic line breaking
breakatwhitespace=false,        % sets if automatic breaks should only happen at whitespace
title=\lstname,                 % show the filename of files included with \lstinputlisting;
                                % also try caption instead of title
escapeinside={\%*}{*)},         % if you want to add a comment within your code
morekeywords={*,...,xchar,xstring,xmany1,chainl1,chainl1',ReadP}            % if you want to add more keywords to the set
}

\newcommand{\theassingment}{Programming Massively Parallel Hardware}
\newcommand{\thesubassignment}{Optimising Tridag}
\newcommand{\shorttheassingment}{PMPH Group Project}
\newcommand{\thepaperauthor}{Morten Espensen \& Niklas Høj \& Mathias Svennson}
\newcommand{\personalid}{dzr440 - 19/06/1991 \& nwv762 - 24/07/1991 \& tpx783 - 20/07/1990}
\newcommand{\thesubject}{Group Project}
\newcommand{\theinstitute}{Department of Computer Science}
\newcommand{\thesupervisor}{Cosmin E. Oancea} %behøves kun hvis findes

\newenvironment{timesfont}{\fontfamily{mathptmx}\selectfont}{}

\newcommand{\n}[0]{\\[2ex]} %NEWLINE [2ex]
\newcommand{\set}[1]{\{#1\}}
\usepackage{hyperref}

\newcommand{\fnurl}[2]{\href{#2}{#1}\footnote{\url{#2}}}  %foodnote url ref \fnurl{String}{Link}

\newcommand{\premise}{\=\mbox{premise}}
\newcommand{\assumption}{\=\mbox{assumption}}
\newcommand{\landi}{\=\intro\land : }
\newcommand{\lande}{\=\elim\land_{1} : }
\newcommand{\landee}{\=\elim\land_{2} : }
\newcommand{\lori}{\=\intro\lor_{1} : }
\newcommand{\lorii}{\=\intro\lor_{2} : }
\newcommand{\lore}{\=\elim\lor : }
\newcommand{\toi}{\=\intro\to : }
\newcommand{\toe}{\=\elim\to : }
\newcommand{\lnoti}{\=\intro\lnot : }
\newcommand{\lnote}{\=\elim\lnot : }
\newcommand{\bote}{\=\elim\bot : }
\newcommand{\lnege}{\=\lnot\elim\lnot : }
\newcommand{\lnegi}{\=\lnot\intro\lnot : }
\newcommand{\leqi}{\= \intro= }
\newcommand{\leqe}{\= \elim= : }
\newcommand{\lforalli}{\= \intro\forall : }
\newcommand{\lforalle}{\= \elim\forall : }
\newcommand{\lexistsi}{\= \intro\exists : }
\newcommand{\lexistse}{\= \elim\exists : }
\newcommand{\MT}{\=MT : }
\newcommand{\PBC}{\=PBC : }
\newcommand{\LEM}{\=LEM : }

\newenvironment{changemargin}[1]{
  \begin{list}{}{
    \setlength{\voffset}{#1}
  }
  \item[]}{\end{list}}


\usepackage{hyperref}
\usepackage{pdfpages} % til at tilføje apendix
\usepackage{graphicx}
\DeclareGraphicsExtensions{.pdf,.png,.jpg}

\def\thesection{\thechapter.\arabic{section}}
%\renewcommand{\thesubsection} {\thesection.\alph{subsection}}

\pagestyle{fancy}
\fancyhead{}
\fancyhead[LO,LE]{\shorttheassingment}
\fancyhead[RO,RE]{\today}
\fancyhead[CO,CE]{\theinstitute}

\begin{document}
\begin{titlepage}
\begin{timesfont}
% Import the ku frontpage graphics
\ThisULCornerWallPaper{1}{nat-farve.pdf}
% Import faculty title
\ThisULCornerWallPaper{1}{nat-en.pdf}
% For forklaring af vspace og stretch, se side 115 i ".. not so short .."
% www.ctan.org/tex-archive/info/lshort/english/lshort.pdf
\vspace*{3cm}
\hspace*{-2.3cm}
% Her vælger en kæmpe skrifttype og fed skrift
\Huge\bfseries
\vspace*{-0.5cm}
\theassingment\n
\hspace*{-2.4cm}
-- \thesubassignment \n
\LARGE
\hspace*{-2.4cm}
\vspace*{-0.5cm}
\thesubject \\[2.2ex]
\hspace*{-2.4cm}
\vspace*{4.3cm}
\theinstitute \n
\hspace*{-2.38cm}
\large
\textbf{Written by:}\\[1ex]
\hspace*{-2.33cm}
\thepaperauthor \\
\hspace*{-2.33cm}
\personalid \\
\hspace*{-2.33cm}
% Nulstiller skriftstørrelse og type (f.eks. fed)
\today
\normalsize
% VEJLEDER INFORMATION I PÅ FORSIDEN?
%\thesubject \n

\vspace*{3.2cm}
\hspace*{-2.33cm}
\textbf{\emph{Supervised by:}}

\hspace*{-2.33cm}
\thesupervisor

\end{timesfont}
\end{titlepage}
\tableofcontents
\newpage

\chapter{Versions of our code}

We have included 6 versions of our code in the attached tarball:

\begin{itemize}
\item The directory \texttt{1_Orig} contains the original code, only
  modified slightly to make e.g. whitespace more consistent with the
  result of our code.
\item The directory \texttt{2_OpenMP} contains our OpenMP code, which
  parallelizes the outer loop in \texttt{run_OrigCPU}.
\item The directory \texttt{3_CUDA} contains our initial CUDA
  version. It is coded without any major code transformations: We have
  simply taking the loops that were naïvely parallelizable and
  implemented kernels for them. To achieve this we inlined the
  \texttt{tridag} function, so parts of it could be made more parallel.
\item The directory \texttt{4_CUDA} gets a large speedup by expanding
  the arrays and moving the outer loop into \texttt{rollback}, thus
  making all the kernels run on a larger number of blocks.
\item The directory \texttt{5_CUDA} gets a further speedup by reducing
  the outer-dimension of the some of the arrays, as their value did not
  depend on that dimension.
\item The directory \texttt{6_CUDA} is our final version. It is for
  all intents and purposes the same as the previous version, except
  almost every memory access have been made coalesced.

\chapter{Transformations}
\section{Stuff 1}
\chapter{Will it validate?}
Yes.
\chapter{Results}
\begin{tabular}{|r|c|c|c|}
  \hline
  Data set size / Implementation & Small & Medium & Large\\
  \hline
  Sequential with flatten arrays & $2162560 \mu s$ & $5652265 \mu s$ & $195845401 \mu s$ \\
 \hline
 OpenMP & $183016 \mu s$ & $241972 \mu s$ & $9680948 \mu s$\\
  \hline
 Naive CUDA & $3956322 \mu s$ & $3639421 \mu s$ & $34980508 \mu s$\\
  \hline
 Optimised CUDA & $183016 \mu s$ & $241972 \mu s$ & $9680948 \mu s$\\
  \hline
\end{tabular}


\end{document}
