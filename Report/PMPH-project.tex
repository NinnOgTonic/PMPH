%Template by Morten Espensen
%Deffinere kommandoer til om vores information

\documentclass[]{report}
%\usepackage{color}
%\usepackage{alltt}
\usepackage{amsmath}
\usepackage{float}
\usepackage[parfill]{parskip}
\usepackage{lscape}
\usepackage{enumerate}

\usepackage[danish,english]{babel}

\usepackage[paper=A4,pagesize]{typearea}
\usepackage[toc,page]{appendix}

\usepackage[utf8]{inputenc}
\usepackage{fancyhdr}
%\usepackage{boxproof}
%\usepackage{daymonthyear}
\usepackage{stmaryrd}

\usepackage{color}
\usepackage{fancyvrb}
\usepackage[usenames,dvipsnames]{xcolor}
\fvset{frame=single,framesep=1mm,fontfamily=courier,fontsize=\scriptsize,numbers=left,framerule=.3mm,numbersep=1mm,commandchars=\\\{\}}

\usepackage{wallpaper} % For the frontpage background

\usepackage{listings} % KODE SHIT
\lstset{ %
language=PHP,          		    % the language of the code
basicstyle=\footnotesize,       % the size of the fonts that are used for the code
numbers=left,                   % where to put the line-numbers
numberstyle=\footnotesize,      % the size of the fonts that are used for the line-numbers
stepnumber=1,                   % the step between two line-numbers. If it's 1, each line
                                % will be numbered
numbersep=5pt,                  % how far the line-numbers are from the code
backgroundcolor=\color{white},  % choose the background color. You must add \usepackage{color}
showspaces=false,               % show spaces adding particular underscores
showstringspaces=false,         % underline spaces within strings
showtabs=false,                 % show tabs within strings adding particular underscores
frame=single,                   % adds a frame around the code
tabsize=2,                      % sets default tabsize to 2 spaces
captionpos=b,                   % sets the caption-position to bottom
breaklines=true,                % sets automatic line breaking
breakatwhitespace=false,        % sets if automatic breaks should only happen at whitespace
title=\lstname,                 % show the filename of files included with \lstinputlisting;
                                % also try caption instead of title
escapeinside={\%*}{*)},         % if you want to add a comment within your code
morekeywords={*,...,xchar,xstring,xmany1,chainl1,chainl1',ReadP}            % if you want to add more keywords to the set
}

\newcommand{\theassingment}{Programming Massively Parallel Hardware}
\newcommand{\thesubassignment}{Optimising Tridag}
\newcommand{\shorttheassingment}{PMPH Group Project}
\newcommand{\thepaperauthor}{Morten Espensen \& Niklas Høj \& Mathias Svennson}
\newcommand{\personalid}{dzr440 - 19/06/1991 \& nwv762 - 24/07/1991 \& tpx783 - 20/07/1990}
\newcommand{\thesubject}{Group Project}
\newcommand{\theinstitute}{Department of Computer Science}
\newcommand{\thesupervisor}{Cosmin E. Oancea} %behøves kun hvis findes

\newenvironment{timesfont}{\fontfamily{mathptmx}\selectfont}{}

\newcommand{\n}[0]{\\[2ex]} %NEWLINE [2ex]
\newcommand{\set}[1]{\{#1\}}
\usepackage{hyperref}

\newcommand{\fnurl}[2]{\href{#2}{#1}\footnote{\url{#2}}}  %foodnote url ref \fnurl{String}{Link}

\newcommand{\premise}{\=\mbox{premise}}
\newcommand{\assumption}{\=\mbox{assumption}}
\newcommand{\landi}{\=\intro\land : }
\newcommand{\lande}{\=\elim\land_{1} : }
\newcommand{\landee}{\=\elim\land_{2} : }
\newcommand{\lori}{\=\intro\lor_{1} : }
\newcommand{\lorii}{\=\intro\lor_{2} : }
\newcommand{\lore}{\=\elim\lor : }
\newcommand{\toi}{\=\intro\to : }
\newcommand{\toe}{\=\elim\to : }
\newcommand{\lnoti}{\=\intro\lnot : }
\newcommand{\lnote}{\=\elim\lnot : }
\newcommand{\bote}{\=\elim\bot : }
\newcommand{\lnege}{\=\lnot\elim\lnot : }
\newcommand{\lnegi}{\=\lnot\intro\lnot : }
\newcommand{\leqi}{\= \intro= }
\newcommand{\leqe}{\= \elim= : }
\newcommand{\lforalli}{\= \intro\forall : }
\newcommand{\lforalle}{\= \elim\forall : }
\newcommand{\lexistsi}{\= \intro\exists : }
\newcommand{\lexistse}{\= \elim\exists : }
\newcommand{\MT}{\=MT : }
\newcommand{\PBC}{\=PBC : }
\newcommand{\LEM}{\=LEM : }

\newenvironment{changemargin}[1]{
  \begin{list}{}{
    \setlength{\voffset}{#1}
  }
  \item[]}{\end{list}}


\usepackage{hyperref}
\usepackage{pdfpages} % til at tilføje apendix
\usepackage{graphicx}
\DeclareGraphicsExtensions{.pdf,.png,.jpg}

\def\thesection{\thechapter.\arabic{section}}
%\renewcommand{\thesubsection} {\thesection.\alph{subsection}}

\pagestyle{fancy}
\fancyhead{}
\fancyhead[LO,LE]{\shorttheassingment}
\fancyhead[RO,RE]{\today}
\fancyhead[CO,CE]{\theinstitute}

\begin{document}
\begin{titlepage}
\begin{timesfont}
% Import the ku frontpage graphics
\ThisULCornerWallPaper{1}{nat-farve.pdf}
% Import faculty title
\ThisULCornerWallPaper{1}{nat-en.pdf}
% For forklaring af vspace og stretch, se side 115 i ".. not so short .."
% www.ctan.org/tex-archive/info/lshort/english/lshort.pdf
\vspace*{3cm}
\hspace*{-2.3cm}
% Her vælger en kæmpe skrifttype og fed skrift
\Huge\bfseries
\vspace*{-0.5cm}
\theassingment\n
\hspace*{-2.4cm}
-- \thesubassignment \n
\LARGE
\hspace*{-2.4cm}
\vspace*{-0.5cm}
\thesubject \\[2.2ex]
\hspace*{-2.4cm}
\vspace*{4.3cm}
\theinstitute \n
\hspace*{-2.38cm}
\large
\textbf{Written by:}\\[1ex]
\hspace*{-2.33cm}
\thepaperauthor \\
\hspace*{-2.33cm}
\personalid \\
\hspace*{-2.33cm}
% Nulstiller skriftstørrelse og type (f.eks. fed)
\today
\normalsize
% VEJLEDER INFORMATION I PÅ FORSIDEN?
%\thesubject \n

\vspace*{3.2cm}
\hspace*{-2.33cm}
\textbf{\emph{Supervised by:}}

\hspace*{-2.33cm}
\thesupervisor

\end{timesfont}
\end{titlepage}
\tableofcontents
\newpage

\chapter{Versions of our code}

We have included 6 versions of our code in the attached tarball:

\begin{itemize}
\item The directory \texttt{1\_Orig} contains the original code, only
  modified slightly to make e.g. whitespace more consistent with the
  result of our code.
\item The directory \texttt{2\_OpenMP} contains our OpenMP code, which
  parallelizes the outer loop in \texttt{run\_OrigCPU}.
\item The directory \texttt{3\_CUDA} contains our initial CUDA
  version. It is coded without any major code transformations: We have
  simply taking the loops that were naïvely parallelizable and
  implemented kernels for them. To achieve this we inlined the
  \texttt{tridag} function, so parts of it could be made more parallel.
\item The directory \texttt{4\_CUDA} gets a large speedup by expanding
  the arrays and moving the outer loop into \texttt{rollback}, thus
  making all the kernels run on a larger number of blocks.
\item The directory \texttt{5\_CUDA} gets a further speedup by reducing
  the outer-dimension of the some of the arrays, as their value did not
  depend on that dimension.
\item The directory \texttt{6\_CUDA} is our final version. It is for
  all intents and purposes the same as the previous version, except
  almost every memory access have been made coalesced.
\end{itemize}

\chapter{Summary of different loops}
There are only two true sources of loop-dependencies in the code:

\begin{itemize}
  \item The loop in \texttt{value} have dependencies upon previous
    versions of the same data.
  \item The loops in \texttt{tridag} are bascially two scans, though
    after the expansions provided in \texttt{3\_CUDA}, we altered it
    to three scans and a few maps, implemented as three kernels.
\end{itemize}

Every other loop is completely parallelizable.

\chapter{Transformations}
\section{Transformation 1 $\rightarrow$ 2: OpenMP privatization}

The transformations done in our OpenMP version are very small: We
moved the calculation of \texttt{PrivGlobs} and \texttt{strike} into
the \texttt{value}-function. This caused the outer loop to be
parallelizable, so we put a pragma on it for OpenMP parallelism.

This caused a ~19 times speedup (from 194.1 seconds to 10.3 seconds).
We did manage to squeeze slightly more performance out of the code,
e.g. by using arrays instead of vectors. However the performance gains
were quite small and the changes quite large (and uninteresting for this
version), so we decided not to include the further optimized version.


\section{Transformation 1 $\rightarrow$ 3: Naïve CUDA}

The purpose of this transformation was to take the original code and
make a parallel version of it without doing much work in terms of code
transformations. We simply took all loops that were trivially
parallizable and created kernels for them.

To make this work without being utterly slow, we also inlined tridag
and unpacked it into a few maps and three scans. The three scans in
each tridag were implemented as 2 kernels.

This change caused a factor 4.6 speedup compared to the previous version
(from around 187.8 seconds to 40 seconds).

\section{Transformation 3 $\rightarrow$ 4: Parallelizing the outer loop}

In this change did not reduce the amount of work to be done or how it
was done. Rather, it made all the kernels run fewer time, but with a
larger grid. This is an improvement, both because the GPU scheduler
in general works better if it gets more work.

Especially some of the tridag-scans work a lot better than before,
because they had so few threads before.

This change caused a factor 5.2 speedup compared to the previous version
(from about 40 seconds to 7.8 seconds).

\section{Transformation 4 $\rightarrow$ 5: Reducing dimensions}

\section{Transformation 5 $\rightarrow$ 6: Coalesced access}

With a working and well-structured naive Cuda implementation, the next step was to optimise it to use coalesced access whenever possbile. To this end, we have written and read from arrays at consecutive indices by consecutive Cuda thread ids whenever possible. In doing this, we have effectively transposed (in-place during assignment) most (if not all) of the arrays from their initial versions.\n
When an array must be arranged one way in one place and transposed in another, we have transposed the array locally in a kernel into shared memory. We did this for kernels 0 and 5, but ended up reversing the change for kernel 5 (more below).\n
To take more advantage of the coalesced access, we tried using various block sizes for the kernels. The block size of kernel 0 remained the same in the tests, namely 32x32, since we use shared memory in this kernel and cannot spawn more threads per core. The table below shows the timings of individual kernels at one-dimensional block sizes 32, 64, 128, 256 and 512.


\begin{figure}[H]
    \centering
\begin{Verbatim}[label={Running time for individual kernels}]
Block size:        32        64       128       256       512

Kernel   0:    149963
Kernel   1:      2927      2517      2516      2497      2455
Kernel   2:     18333     18504     18582     18622     18644
Kernel   3:     93541     53443     36580     37026     37070
Kernel   4:    132734     96662     85957     91119     92739
Kernel   5:    143290     86995     70856     71047     71148
Kernel   6:      2951      2478      2408      2412      2351
Kernel   7:     18178     18291     18385     18446     18471
Kernel   8:    166507    154633    153778    154123    154369
Kernel   9:    132842     98206     87702     91106     91275
Kernel  10:      7143      6474      6470      6523      6537

Total time:   2031194   1860382   1793703   1808618   1818025
\end{Verbatim}
\caption{Table of performance in micro-seconds of each kernel}
\end{figure}

Looking at the table, we found that all kernels (besides 0) performed best (allowing room for variation in measurements) at block size 128.\n
Kernel 5 remains partially non-coalesced, and we did try to fix that by using shared memory and a block size of 32x32. However, the resulting kernel took about 110000 micro-seconds to execute, which was an improvement to the 143290 micro-seconds from the one-dimensional block size of 32. But the one-dimensional block size of 128 without shared memory only took 70856 microseconds, which is clearly better.\n
We suspect that a similar improvement might be made in kernel 0, again by sacrificing shared memory for a one-dimensional block size of 128 (or others), but we will not pursue this optimisation this time around.\n
Looking at the kernel timings as above, we continued to tweak on coalesced accesses, benefiting in one kernel at the cost of another, searching for the optimal configuration.\n
The final kernel timings are listed below, and the final running time was 1695149 micro-seconds (1.7 seconds).\n


\begin{figure}[H]
    \centering
\begin{tabular}{| c | c |}
    \hline
Kernel & micro-seconds \\
    \hline
6  & 2214 \\
1  & 2278 \\
10 & 5961 \\
7  & 18728 \\
2  & 18760 \\
3  & 36609 \\
8  & 37111 \\
9  & 85748 \\
4  & 85761 \\
5  & 96671 \\
0  & 156493 \\
    \hline
\end{tabular}
\caption{Kernel timings after all optimisations}
\end{figure}

It should be noted that all timings in this section were taken October 28th, but the same program on the same server now runs significantly faster on the 31st. Before, the optimal version took 1.7 seconds, but now the same version takes 0.79 seconds.\n

These optimisations caused a factor 4.8 speed-up compared to the previous version (from about 3.9 seconds to 0.79 seconds).




\chapter{Does it validate?}
Yes.
\chapter{Results}
\begin{tabular}{|r|c|c|c|}
  \hline
  Data set size / Implementation & Small & Medium & Large\\
  \hline
  Sequential with flatten arrays & $2162560 \mu s$ & $5652265 \mu s$ & $195845401 \mu s$ \\
 \hline
 OpenMP & $183016 \mu s$ & $241972 \mu s$ & $9680948 \mu s$\\
  \hline
 Naive CUDA & $3956322 \mu s$ & $3639421 \mu s$ & $34980508 \mu s$\\
  \hline
 Optimised CUDA & $183016 \mu s$ & $241972 \mu s$ & $9680948 \mu s$\\
  \hline
\end{tabular}


\end{document}
